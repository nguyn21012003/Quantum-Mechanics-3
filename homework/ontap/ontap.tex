\documentclass{article}
\usepackage[utf8]{vietnam}
\usepackage[utf8]{inputenc}
\usepackage{anyfontsize,fontsize}
\changefontsize[13pt]{13pt}
\usepackage{commath}
\usepackage[d]{esvect}
\usepackage{parskip}
\usepackage{xcolor}
\usepackage{amssymb}
\usepackage{slashed,cancel}
\usepackage{indentfirst}
\usepackage{pdfpages}
\usepackage{graphicx}
\usepackage{upgreek}
\usepackage{nccmath,nicematrix}
\usepackage{mathtools}
\usepackage{amsfonts,float}
\usepackage{amsmath,systeme}
\usepackage[thinc]{esdiff}
\usepackage{hyperref}
\usepackage{bm,physics}
\usepackage{fancyhdr}
%footnote
\pagestyle{fancy}
\renewcommand{\headrulewidth}{0pt}%
\fancyhf{}%
\fancyfoot[L]{Vật lý Lý thuyết}%
\fancyfoot[C]{\hspace{6.5cm} \thepage}%


\usepackage{geometry}
\geometry{
	a4paper,
	total={170mm,257mm},
	left=20mm,
	top=20mm,
}


\newcommand{\image}[1]{
	\begin{figure}[H]
		\centering
		\includegraphics[width=0.75\textwidth,height=5.0cm]{pic/#1}
		\label{#1}
	\end{figure}
}
\renewcommand{\l}{\ell}
\newcommand{\dps}{\displaystyle}
\newcommand{\mean}[1]{\langle{#1}\rangle}
\newcommand{\f}[2]{\dfrac{#1}{#2}}
\newcommand{\at}[2]{\bigg\rvert_{#1}^{#2} }


\renewcommand{\baselinestretch}{2.0}


\title{\Huge{Ôn tập cơ lượng tử 3}}

\hypersetup{
	colorlinks=true,
	linkcolor=black,
	filecolor=magenta,      
	urlcolor=cyan,
	pdftitle={QM3},
	pdfpagemode=FullScreen,
}

\urlstyle{same}

\begin{document}
\setlength{\parindent}{20pt}
\newpage
\author{TRẦN KHÔI NGUYÊN \\ VẬT LÝ LÝ THUYẾT}
\maketitle
\section{Bài tập chương 6}	
\subsection*{6.28}
Khai triển Taylor cho $\Psi(x,t)$ ta có
\begin{equation*}
	\begin{aligned}
		\Psi(x,t_{0} + \delta) 
		&= \Psi(x,t_{0}) + \delta \f{\partial \Psi}{\partial t} \at{t_{0}}{} + ... \\
		&= \Psi(x, t_{0}) + \delta \f{1}{i \hbar} \hat{H}(t_{0}) \Psi(x,t_{0})\\
		&= \left[ 1 - \f{i \delta}{\hbar} \hat{H}(t_{0}) + ... \right] \Psi(x,t_{0})
	\end{aligned}
\end{equation*}	
\section{Bài tập chương 11}
\subsection*{11.18}
\image{11.18.png}
\begin{enumerate}
	\item [(a)] 
	Năng lượng 
	\begin{gather*}
		E_{n} = \f{n^{2} \pi^{2} \hbar^{2}}{2ma^{2}}; \quad \psi(x,0) = \sqrt{\f{2}{a}} \sin \f{\pi x}{a} 
	\end{gather*}  
	Khi tăng bề rộng giếng lên gấp đôi ngay lập tức
	\begin{gather*}
		E_{n} = \f{n^{2} \pi^{2} \hbar^{2}}{2m(2a)^{2}}; \quad \psi_{n}(x,0) = \sqrt{\f{2}{2a}} \sin \f{n\pi x}{2a}
	\end{gather*}
	Ta khai triển hàm sóng ban đầu theo các $\psi_{n}$
	\begin{gather*}
		\psi(x,t) = \sum_{n=1}^{\infty}c_{n} \psi_{n}(x)
	\end{gather*}
	Ta đi tính các hệ số $c_{n}$ cho $\left[0,a\right]$
	\begin{equation*}
		\begin{aligned}
			c_{n}
			&= \f{\sqrt{2}}{a} \int_{0}^{a} \sin\left(\f{\pi x}{a}\right) \sin \left(\f{n\pi x}{2a}\right) dx\\
			&= \f{\sqrt{2}}{2a} \int_{0}^{a} \cos \left[\left(\f{n}{2} - 1\right)\f{\pi x}{a}\right] + \cos \left[\left(\f{n}{2} + 1\right)\f{\pi x}{a}\right]\\
			&= \f{4\sqrt{2}}{\pi} \f{\sin\left[(\f{n}{2} + 1)\pi\right]}{n^{2 - 4}}
			=
			\begin{cases}
				0 \quad if \; n \; even\\
				\pm \f{4\sqrt{2}}{\pi(n^{2} - 4)} \quad if \; n \; odd
			\end{cases}
		\end{aligned}
	\end{equation*}
	Ta có được xác suất 
	\begin{gather*}
		P_{n} = \abs{c_{n}}^{2}
		\begin{cases}
			\f{1}{2} \; n = 2\\
			\f{32}{\pi^{2} (n^{2} - 4)} \; n \; odd
		\end{cases}
	\end{gather*}
\end{enumerate}
\subsection*{11.20}
\image{11.19.png}
Năng lượng ban đầu 
\begin{gather*}
	E_{n} = (n + 1/2) \hbar \omega
\end{gather*}
Năng lượng khi tăng tần số $\omega' = 2 \omega$
\begin{gather*}
	E_{n}^{'} = 2(n + 1/2) \hbar \omega
\end{gather*}
Vậy xác suất để đo được năng lượng $E_{n} = \f{1}{2}\hbar \omega$ là bằng 0. Xác suất để đo được $E_{n} = \hbar \omega$
















































































































































































































































































































































































































































































































































































































































































































































































































































	
\end{document}